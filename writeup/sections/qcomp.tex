\section{Kvanteberegninger}

\subsection{Intro til kvantemekanikk}

    \subsubsection*{Notasjon}

        La $\mathcal{H}$ være et Hilbertrom, vi definerer $\partial D(\mathcal{H})=\{v:\mathcal{H}\mid ||v||=1\}$ til å være randen av enhetskulen i $\mathcal{H}$ og $U(\mathcal{H})$ til å være mengden av unitære operatorer på $\mathcal{H}$. I fysikk bruker man en ket $|\_\rangle$ for å utheve at et element i keten er en vektor, på samme måte som pil notasjon for vektorer. En bra $\langle\_ |$ betegner det elementet i dualrommet som gjør at $\langle\psi |\psi\rangle = ||\psi||^2$. Dette er veldefinert fra Riesz-Frechet representasjons teorem.

    \subsubsection*{Postulatene i kvante}

        Kvantemekanikk er en beskrivelse av fysiske systemer på små størrelser. Reglene for hvordan disse systemene oppfører seg er formulert utifra 6 postulater. \cite{Jaffe_2007} og \cite{portugal_2019} definerer postulatene som følger
        \begin{enumerate}
            \item På hvert øyeblikk er tilstanden til det fysiske systemet beskrevet av en ket $|\psi\rangle$ i rommet av tilstander.
            \item Enhver observabel fysisk egenskap av systemet er beskrevet av en operator som virker på ketten som beskriver systemet.
            \item De eneste mulige resultatene av en måling av en observabel $\mathcal{A}$ er egenverdiene til den assosierte operatoren $A$.
            \item Når en måling er gjort på en tilstand $|\psi\rangle$ er sannsynligheten for å få en egenverdi $a_n$ gitt ved kvadratet av indreproduktet til $|\psi\rangle$ sammen med egenprojeksjonen $P_{a_n}$.
            \begin{align*}
                p_{a_n} = \langle \psi|P_{a_n}|\psi\rangle
            \end{align*}
            \item Umiddelbart etter en måling av en observabel $\mathcal{A}$ har gitt egenverdien $a_n$, er systemet i tilstanden til den normaliserte egenprojeksjonen $P_{a_n}|\psi\rangle$.
            \item Tidsutviklingen til et system bevarer normen til en ket $|\psi\rangle$.
        \end{enumerate}

        For å forklare postulatene sammen med et eksempel antar vi at det finnes en partikkel som har to observable tilstander, vi kaller de spin opp og spin ned, som er beskrevet av vektorer i et rom av tilstander. Her tolkes rommet av tilstander som et (kompleks separabelt) Hilbertrom. En tilstand eller ket $|\psi\rangle$ er dermed en vektor i $\mathcal{H}$. Siden $\mathcal{H}$ er et Hilbertrom har den også en basis $\{|\beta_\lambda\rangle\mid\lambda : \Lambda\}$, hvor $\Lambda$ er en indeksmengde. Ettersom vi har to observable tilstander holder det å anta at $\mathcal{H}=\mathbb{C}^2$ og at $|\beta_i\rangle = e_i$ for $i=1,2$. Man kan skrive $|\psi\rangle$ som en lineærkombinasjon av basisen, dette er også kalt for en superposisjon av tilstandene $\{|\beta_\lambda\rangle\mid\lambda : \Lambda\}$ (eller $\{e_1, e_2\}$ for spin eksemplet).
        \begin{align*}
            |\psi\rangle & = \Sigma\psi_i|\beta_i\rangle \\
            (|\psi\rangle & = \psi_1e_1 + \psi_2e_2)
        \end{align*}

        Gitt at vi har en observable $\mathcal{A}$, altså en fysisk egenskap ved systemet som kan måles, så vet vi at dette er gitt ved en lineærtransformasjon $A$ som virker på Hilbertrommet $\mathcal{H}$. De fysiske målingene til systemet skal være gitt ved egenverdiene av denne lineærtransformasjonen, noe som krever den til å være en endomorfi, aka. $A:\mathcal{H}\rightarrow\mathcal{H}$. I spin eksemplet kan man måle spin opp og spin ned med en observable hvor f.eks. spin opp har egenverdien 1 og spin ned har egenverdien -1. En vanlig antagelse er at alle de observable egenskapene skal være reelle verdier, derfor velger man i tillegg å anta at $A$ må være en Hermitisk operator (selvadjungert).

        Når man gjør en måling av en observabel er det tilfeldig hva man måler. Sannsynlighetene for å måle de forskjellige egenverdiene er gitt ved formelen over. Etter måling vil systemet kollapse ned i egenrommet til vektoren $\frac{1}{\sqrt{p_{a_n}}}P_{a_n}|\psi\rangle$. Hvis den algebraiske multiplisiteten til egenverdien $a_n$ er $1$ så tilsvarer dette vektoren $\frac{|a_n\rangle}{|||a_n\rangle ||}$. En konsekvens av dette er at observabelen som har tilstanden $|\psi\rangle$ som en egenvektor med egenverdi lik $1$ vil ikke endre tilstanden til systemet etter måling. Hvis man derimot måler denne observabelen, vil man normalisere tilstanden. I spin eksemplet vil dette si at hvis man måler verdien 1, så vil tilstanden kollapse til den tilsvarende egenvektoren, normalisert. En konsekvens av dette er at en tilstand er bedre definert som ekvivalensklasser langs linjer i Hilbertrommet. En tilstand er dermed et element i randen av enhetskulen til Hilbertrommet.
        \begin{align*}
            |\psi\rangle : \partial D(\mathcal{H}) = \{v:\mathcal{H}\mid ||v||=1\}
        \end{align*}

        Det siste postulatet forteller oss hvordan et system utvikler seg. Formelen $|\psi (t)\rangle = U(t,t_0)|\psi (t_0)\rangle$ brukes ofte for å beskrive hvordan dette ser ut. Siden vi krever at $U(t,t_0)$ skal bevare normen til $|\psi (t_0)\rangle$, dvs. at det finnes en virkning $U(t,t_0):\partial D(\mathcal{H})\rightarrow\partial D(\mathcal{H})$, følger det at denne operatoren er unitær. Mengden $U(\mathcal{H})$ vil betegne de unitære operatorene som operer på det Hilbertrommet. For vårt formål kan man tenke på et kvantesystem i $U(\mathcal{H})$-mengden $\partial D(\mathcal{H})$.

        Som beskrevet av \cite{portugal_2019} kan man lage kompositter av kvantesystemer med det algebraiske tensorproduktet. Gitt to foskjellige kvantesystemer beskrevet av to forskjellige Hilbertrom $\mathcal{H}_1$ og $\mathcal{H}_2$ så kan man lage rommet av sammensatte tilstander som $\mathcal{H}_1\otimes\mathcal{H}_2$. Vi får da en klasse med observable og en klasse med operatorer som handler på systemene gjennom tensorproduktet. Man kan vise at tensorproduktet av to unitære og hermitiske matriser er igjen unitære og hermitiske, det er derfor veldefinert å betrakte tensoren for kompositt systemer. Sammenfiltringsfenomenet foregår når man konstruerer slike kompositt systemer. Hvis vi antar at vi har to partikler med spin egenskapen $\phi$ og $\psi$ og komposittsystemet $|\phi\psi\rangle = \sigma_0e_0\otimes e_0 + \sigma_1e_1\otimes e_1$, så vil en måling av den ene partikkelen ende opp med å måle den andre partikkelen. Dersom man måler egenverdien til $e_0$ for $\phi$ så vil systemet kollapse til $\frac{1}{\sigma_0}P_1\otimes I(|\phi\psi\rangle) = e_0\otimes e_0$. Dette medfører at alle målinger av $\psi$ vil gi egenverdien $1$.

\subsection{Qubits og kvantekretser}
    
        Klassiske bits har to tilstander: $0$ eller $1$. Kvantebits, eller qubits er et fysisk system som har en observabel som måler to diskrete tilstander. Disse tilstandene bruker vi for å representere $0$ og $1$. Ettersom at operatoren som måler $0$ og $1$ er hermitisk, så finnes det en ortonormal basis for Hilbertrommet som diagonaliserer denne operatoren. Elementene i denne basisen vil bli betegnet som $|0\rangle$ og $|1\rangle$. En qubit $q$ er dermed et element i $\partial D(\mathbb{C}^2)$ på formen $q = q_0|0\rangle + q_1|1\rangle$. 

        En streng av bits er sammensettingen av flere bits. På samme måte konstruerer vi en streng av qubits til å være sammensettingen av flere qubits. Denne sammensettingen er gitt av tensorproduktet mellom de algebraiske qubitsene. F.eks. er en 2-qubit streng er et element i $\partial D(\mathbb{C}^2\otimes\mathbb{C}^2)$ på formen under.
        \begin{align*}
            q = q_{00}|0\rangle\otimes |0\rangle + q_{01}|0\rangle\otimes |1\rangle + q_{10}|1\rangle\otimes|0\rangle + q_{11}|1\rangle\otimes |1\rangle
        \end{align*}
        For kortfatthetens skyld skriver vi $|ab\rangle = |a\rangle\otimes |b\rangle$. Notasjonen $|\_\rangle$ vil få en ekstra presisjon i denne rapporten. La $nBit$ være mengden av strenger med n-bits, vi definerer $|\_\rangle : \bigcup_{n=0}^{\infty}nBits \rightarrow \mathbb{T}(\mathbb{C}^2)$ til å være en funksjon fra alle strenger og inn i tensoralgebraen til $\mathbb{C}^2$. Den er definert på $|0\rangle$ og $|1\rangle$ som over, også utvides den lineært og fritt over tensoralgebraen. 

        % Et eksempel på hvordan disse qubitene oppfører seg i naturen er via EPR (Einstein, Podolsky og Rosen) paret. Vi sier at et system av qubits er sammenfiltret hvis det ikke kan skrives som en elementær tensor, altså på formen. Et EPR par er et 2-qubit system på formen $\psi = \frac{1}{\sqrt{2}}|00\rangle + \frac{1}{\sqrt{2}}|11\rangle$. Man kan se at dette systemet er sammenfiltret, ettersom de to elementære tensorene ikke har noen felles faktorer. Hvis vi derimot måler den første qubiten i systemet vil vi ende opp med at det er en $(\frac{1}{\sqrt{2}})^2 = 50\%$ sjanse for å måle $0$ og $50\%$ sjanse for å måle $1$. Hvis vi derimot har målt $0$ på den første qubiten, så vil systemet kollapse til $\psi = |00\rangle$, og vi vet dermed at den andre qubiten må være i tilstand $|0\rangle$. \todo[color=yellow]{Legg til mer formalitet og kom med flere eksempler etterpå?}

\subsection{Kvantealgoritmer og orakler}

        En klassisk algoritme kan sees på å være en metode som løser et problem basert på et input av bits. Ofte representerer disse bitsa andre datastrukturer som tall, grafer eller flytnettverk. En kvantealgoritme kan dermed sees på å være en metode som løser et problem basert på et input av qubits. I mange tilfeller vil dette inputtet være operasjoner på et kvantesystem. Selve algoritmen er en komposisjon av unitære operatorer og målinger som opererer på en tilstand $\psi$ i $\partial D(\mathcal{H})$. 
        
        Kvantekretser kan brukes til å representere kvantealgoritmer. Disse kretsene består av registre, arbeidsplass, logiske kvanteporter og målinger. Registeret er de qubitsene som man har som input i programmet og arbeidsplassen er ekstra qubits som man kan bruke for å gjøre operasjoner på registeret med. En logisk kvanteport er en port som utfører en unitær operasjon på en eller flere qubits. En måling måler en egenverdi til kvantesystemet og kollapser gir tilstanden som en bit. Disse elementene komponerer man i et flytdiagram for å danne en kvantekrets. Her er et eksempel på en slik krets\todo[color=magenta]{Bilde av kvantekrets her}

        De simpleste logiske kvanteportene som vi har er de som operer på 1 qubit, dette er de unitære matrisene over $\mathbb{C}^2$. Av de mest kjente så har vi matrisene:
        \begin{center}
            \begin{math}
                I = \begin{pmatrix}
                    1 & 0 \\ 0 & 1
                \end{pmatrix},\ X = \begin{pmatrix}
                    0 & 1 \\ 1 & 0
                \end{pmatrix},\ Y = \begin{pmatrix}
                    0 & -i \\ i & 0
                \end{pmatrix},\ Z = \begin{pmatrix}
                    1 & 0 \\ 0 & -1
                \end{pmatrix},\ R_\theta = \begin{pmatrix}
                    1 & 0 \\ 0 & e^{i\theta}
                \end{pmatrix} og\ H = \frac{1}{\sqrt{2}}\begin{pmatrix}
                    1 & 1 \\ 1 & 1
                \end{pmatrix}
            \end{math}
        \end{center}

        $I$, $X$, $Y$ og $Z$ matrisene kalles for Pauli matriser, $R_\theta$ kalles for en phaseskift matrise og $H$ kalles for Hadamard matrisen. To andre viktige porter er CNOT og SWAP, som operer på 2-qubits. CNOT, også kalt for "controlled not" operer på 2-qubits ved å anvende $X$ matrisen på den andre qubiten i registeret hvis den første qubiten har verdi 1. SWAP bytter om på rekkefølgen til qubitene.
        \begin{center}
            \begin{math}
                CNOT = \begin{pmatrix}
                    1 & 0 & 0 & 0 \\
                    0 & 1 & 0 & 0 \\
                    0 & 0 & 0 & 1 \\
                    0 & 0 & 1 & 0
                \end{pmatrix},\ SWAP = \begin{pmatrix}
                    1 & 0 & 0 & 0 \\
                    0 & 0 & 1 & 0 \\
                    0 & 1 & 0 & 0 \\
                    0 & 0 & 0 & 1
                \end{pmatrix}
            \end{math}
        \end{center}\todo[color = magenta]{Tegn kretser av hvordan disse virker}
        \todo[color=yellow]{Universalitet av kvante kretser?}

        Skriv noe om merkeorakel og faseorakel. Hvorfor trenger vi de, og hva brukes de til? Hva er problemet når det kommer til orakler?