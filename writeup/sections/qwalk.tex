\section{Kvantevandringer}

\subsection{Grover's algoritme og metoden av amplitude amplifikasjon}

    Grover's algoritme løser et veldig spesifikt problem, men teknikken som den bruker for å løse problemet er veldig interessant å studere. Problemet går som følger: Tenk at man er gitt en bistring med $N=2^n$ bits, hvor $t$ bits er satt til 1. Finn minst 1 bit som har verdi 1. Dette problemet kan åpenbart løses i worst case lineær tid med konstant minne ved å randomisert iterere gjennom alle bitene og sjekke om de er 1 eller 0. Om den er 1 kan man terminere programmet, og returnere den posisjon som ga 1. Grover's algoritme har en kvadratisk hastighetsøkning ved å manipulere superposisjoner av qbits.

    For å beskrive problemet med et fysisk kvante system trenger vi å oversette problemet først. La $(b_k)_N$ være bitstringen med lengde N, definer så orakelet $\mathcal{O}_{(b_k)_N}:\mathbb{C}^{2^n} \otimes \mathbb{C}^2 \rightarrow \mathbb{C}^{2^n} \otimes \mathbb{C}^2$ til å merke målbiten hvis registerbiten var en løsning. Dette vil si at hvis $\mathcal{O}_{(b_k)_N}(|r\rangle\otimes |0\rangle) = |r\rangle\otimes |1\rangle$ så følger det at $b_r = 1$. For å fullføre Grover's algoritme trenger vi en ny matrise til, og det er en matrise som flipper fortegnet til registeret hvis den ikke er tilstanden $|0\rangle^{\otimes n}$.
    \begin{center}
        \begin{math}
            R = \begin{pmatrix}
                1 & 0 & ... \\
                0 & -1 & ... \\
                \vdots & \vdots & \ddots
            \end{pmatrix}
        \end{math}
    \end{center}

    En Grover iterate $\mathcal{G}$ er definert som $\mathcal{G}=H^{\otimes n}RH^{\otimes n}\mathcal{O}_{(b_k)_N}$. Algoritmen Gorver's algoritme er definert som $G = M\mathcal{G}^k\circ H^{\otimes n}$, hvor $M$ er en projektiv måling, og $k$ er en konstant. Man skal da kunne fastslå med en høy sannsynlighet at målingen gir deg posisjonen til en bit i $(b_k)_N$ som er $1$. For å finne denne $k$-en som man bruker for å kjøre algoritmen trenger vi se på metoden av amplitude amplifikasjon.

    \lipsum[7]
        
\subsection{Kvantevandringer basert på kvantemyntkast}

\lipsum[8]

\lipsum[9]

\lipsum[10]

\subsection{Kvantesøk}

\lipsum[11]