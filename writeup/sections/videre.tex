\section{Videre lesning}

    \subsubsection*{Szegedy vandring og andre rammeverk}
    
        En annen form for kvantevandring som ikke har blitt diskutert i denne rapporten, men som er svært omtalt er Szegedy vandringer \cite{dewolf2021quantum}\cite{Venegas_Andraca_2012}\cite{portugal_2019}. Ideen med Szegedy vandringer er å gjøre om klassiske tilfeldige vandringer basert på Markov kjeder til en kvantevandring. Før Szegedy foreslo sin definisjon for en kvantevandring var det fremdeles uklart hvordan man vandret over generelle grafer. Teknikken han bruker er en transformasjon av transisjonsmatrisen til en Markov kjede. Denne transformasjon går ut på å lage to projeksjonsmatriser, og bruke de til å lage en refleksjon om rommet de projiserer ned på. Deretter finner man kvantevandringsoperatoren som en komposisjon av disse matrisene. 
        
        Kvantevandringen som er beskrevet av Szegedy er en kvantevandring som foregår over bipartitte grafer. Metoden kan generelt beskrives over slike grafer, og metoden for å transformere transisjonsmatrisen burde tilsvare en transformasjon av grafen til en bipartitt graf.
        
        Siden Szegedy har mange andre former for kvantevandringer blitt foreslått. De kvantevandringene som har blitt sett på i denne rapporten tilhører nok klassen av kvantevandringer kalt "hitting time based quantum walks". Blant disse nye metodene for å lage kvantevandringer, mangles det et godt sammenligningsgrunnlag. Noe av dagens forskning foregår ved å prøve å forene slike kvantevandringer \cite{apers2019unified}.

    \subsubsection*{Elektriske nettverk}
    
        I sannsynlighetsteori og tilfeldige vandringer finner man elektriske nettverk. Et elektrisk nettverk er et nettverk generert fra en vektet graf og en initiell betingelse. Elektriske nettverk har funnet anvendelser i tilfeldige vandringer og de gir et mål på den klassiske "hitting time" definisjonen. Vektene på grafen representerer sjansen for at en tilfeldig vandrer vil bevege seg langs den kanten. I et elektrisk nettverk retter vi hver kant og gir hver rettede kant en resistanse. Denne resistansen baserer seg normalt sett på en kvotient av vekten til kanten og den totale vekten i grafen.
        
        I artikkelen til Apers \cite{apers2019unified} bruker han elektriske nettverk for å definere kvantevandringer og finne mål på hvor effektive disse algoritmene er.
    
    \subsubsection*{Cayley grafer og Johnson grafer}

        Å studere grafer med mer struktur har gitt rik teori til kvantevandringer. Kvantevandringer over Cayley grafer har blitt blant annet studert av \cite{dai2018discretetime}. En Cayley graf er en graf som har blit generert av en Gruppe. La $H$ være en gruppe med generatorer $\{h_1,...,h_d\}$, vi definerer en rettet graf $G=(V,E)$ til å ha noder $V=H$ og kanter fra $g_1$ til $g_2$ hvis det finnes en $h_i$ slik at $g_2 = g_1h_i$.
        
        En kvantevandring som vi kan definere her er en position-coin notation vandring, med $\mathcal{H}_V=\mathbb{C}V$ og $\mathcal{H}_C=\mathbb{C}^d$. Siden vi ikke kan garantere at det minimale kant kromatiske tallet er $d$ virker det ikke generelt at shift operatoren er flip-flop operatoren. For alle Cayley grafer vet vi at undergrafen spent av kantene definert ved $h_1$ definerer en disjunkt union av sykliske grafer. Shift operatoren kan defineres på følgende måte:
        \begin{align*}
            S(|g\rangle,|i\rangle) = |gh_i\rangle
        \end{align*}
        
        Denne kvantevandringen virker mer generelt, og kan brukes for $d$-regulære grafer. Vi krever i dette tilfellet at vi transformer grafen til en rettet graf, ved å lage to piler for enhver kant. Det følger da at grafen kan dekomponeres ned til $d$ undergrafer som er en disjunkt union av sykliske grafer.
        
        Disse kvantevandringene kan man nok bruke til å studere egenskaper ved Cayley grafer og enkelte Johnson grafer. Ved å i tillegg implementere en kvante adder for gruppen man vandrer over, kan man abstrahere vekk de relevante beregningene som kreves å gjøre av gruppen.
    
    \subsubsection{Kontinuerlig tid kvantevandringer}
    
        I denne rapporten har vi kun begrenset oss til diskret tid kvantevandringer. Dette er en bare en liten klasse av kvantevandringer som finnes. Ved å bruke Schrodingers likning kan man definere kontinuerlig tid kvantevandringer. Her er utfordringen å definere den Hamiltonske operatoren som brukes. Venagas-Andraca \cite{Venegas_Andraca_2012} beskriver en type av disse kontinuerlig tid kvantevandringene.
        
        I dag virker det som om denne klassen av kvantevandringer har mer bluss. Noen resultater om denne klassen er at det er vist at de kan gi eksponesiell algoritmisk speed-up, istedenfor kvadratisk. Dette skjer som oftest i spesielle tilfelle av kvantevandringer og om man kan speede-up flere algoritmer er av stor interesse.
        
        Skillet mellom DTQW (diskrete) og CTQW har lenge vært vanskelig å se. Det er ganske nylig at noen har klart å transformere en CTQW algoritme til en DTQW algoritme og vice versa. Dette er kanskje overraskende at det ikke er en tydelig sammenheng mellom disse klassene av algoritmer.
        
    \subsubsection*{Simulering av kvantesystmer}
    
        Kvantevandringer har funnet anvendelsen til å simulere arbitrære kvantesystmer. I artikkelen til Venagas-Andreca \cite{Venegas_Andraca_2012} så beskriver de en modell for hvordan kvantevandringer kan brukes til å simulere kvanteberegninger. Disse teknikkene kan nok klare å finne anvendelser til å simulere fysiske og kjemiske kvantesystemer.